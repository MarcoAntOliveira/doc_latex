\documentclass[12pt]{article}
\usepackage[portuguese]{babel}
\usepackage[utf8]{inputenc}
\usepackage[T1]{fontenc}
\usepackage{color}
\usepackage{lipsum}

 
\author{Marcos antonio}
\title{Transformada de Laplace - Sinais e sistemas}
\date{\today}

\begin{document}
\maketitle
\begin{abstract}
    O presente trabalho visa apresentar um breve resumo sobre a transformada de laplace para sistemas causais. 
\end{abstract}
\tableofcontents

\section{Introdução}
Para um sinal x(t), a transformada de Laplace L{x(t)} = X(s) é definida por:
\[\textcolor{blue}{X(S) = \int_{-\infty}^{\infty} x(t)e^{-st}   \, dt}\quad onde\quad s=\sigma +j\omega 
\quad \textcolor{blue}{X(s)= \int_{-\infty}^{\infty}x(t)e^{-(\sigma +j\omega)}dt = \int_{0}^{\infty}x(t)e^{-\sigma t}dt}\] Esta equação é conhecida como transformada de Laplace bilateral.
\section{Transformada de Laplace para Sistemas causais}
\[\textcolor{blue}{rL\{x(t)u(t)\} = X(S)= \int_{-\infty}^{\infty} x(t)u(t)e^{-st} \, dt = \int_{0^{+}}^{\infty} x(t)e^{-st}  \, dt}\]A existência de X(s) está condicionada à convergência da integral da definição, ou seja, que integral seja absolutamente integrável.
\[\textcolor{blue}{ \int_{-\infty}^{\infty} |f(t)| dt < \infty }\]
\section{Convergência}
Portanto, a transformada de Laplace converge se:
\[\textcolor{blue}{or\int_{-\infty}^{\infty} |x(t)e^{-\sigma t}dt| <{\infty} \quad converge\quad se\quad \sigma > 0 }\]
Desta forma, X(s) existe somente para valores de Re{s} localizados numa região do plano complexo denominada Região de Convergência (ROC – region of convergence)
\section{Região de Convergência (ROC – region of convergence)}
A ROC é necessária para a determinação da transformada inversa de Laplace x(t) de X(s), definida pela equação ao lado:
\[\textcolor{blue}{x(t) =\frac{1}{2\pi j} \int_{c-j\infty}^{c +j\infty} X(S)e^{st}ds}\]
onde c é uma constante escolhida para garantir a convergência da integral. Porém, nós iremos utilizar tabelas para calcular x(t) ao invés de fazer uso da integral acima.
\footnote{Este texto feito para realizar o estudo de tal assunto}\\\\\\\\



\begin{minipage}{0.4\textwidth}
    \centering
    \hrule
    Assinatura do Autor
\end{minipage}
\end{document}
