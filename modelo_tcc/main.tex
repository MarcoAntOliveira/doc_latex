\documentclass[
	% -- opções da classe memoir --
	12pt,				% tamanho da fonte
	%openright,			% capítulos começam em pág ímpar (insere página vazia caso preciso)
	oneside,			% para impressão no anverso. Oposto a twoside
	a4paper,			% tamanho do papel. 
        % hyphens,            % corrigir hifenização com urls longas.
	% -- opções da classe abntex2 --
	chapter=TITLE,		% títulos de capítulos convertidos em letras maiúsculas
	section=TITLE,		% títulos de seções convertidos em letras maiúsculas
	%subsection=TITLE,	% títulos de subseções convertidos em letras maiúsculas
	%subsubsection=TITLE,% títulos de subsubseções convertidos em letras maiúsculas
	% -- opções do pacote babel --
	english,			% idioma adicional para hifenização
	% french,			% idioma adicional para hifenização
	%spanish,			% idioma adicional para hifenização
	brazil,				% o último idioma é o principal do documento
	% french
	]{abntex2}
 
\usepackage{tccctj} % carrega o estilo CTJ
% \usepackage[semdicas]{tccctj} % suprimir dicas do template

\bibliographystyle{abntex2-alf-ufsc} % Arquivo bst com patch para correção de citação de proceedings. Produz italico em "In", conforme descrito em: https://github.com/abntex/abntex2/issues/226

% ----------------------------------------------------
% Pacotes adicionais
\usepackage{graphicx}
\usepackage{multicol}
\usepackage{multirow}
\usepackage{tabularx}
\usepackage{quoting}
\quotingsetup{indentfirst={false},font={footnotesize},leftmargin=4cm,rightmargin=0cm} 
\hypersetup{hidelinks}
\usepackage{enumitem}
\setlist{noitemsep}

% ----------------------------------------------------

% ----------------------------------------------------
% Informações do trabalho
\autor{Nome do aluno}
\titulo{Título do TCC}
% \subtitulo{Subtítulo} % Preenchimento opcional para quando houver sub-título
\orientador{Nome do orientador}
\coorientador{Nome do coorientador} % Preenchimento opcional para quando houver co-orientador
\curso{Engenharia X}
% \titulacao{Bacharel em x}
\datadadefesa{23}{fevereiro}{2023}
\local{Joinville}
% ----------------------------------------------------

\begin{document}

% Elementos pré-textuais

% 1. Capa do trabalho
\imprimircapa

% 2. Folha de rosto
\imprimirfolhaderosto

% 3. Folha de aprovação
\begin{folhadeaprovacao}
    \membrodabanca[Orientador/Presidente]{Prof. Dr. Nome do Orientador/Presidente}{}
    \membrodabanca{Prof. Dr. Membro da banca 1}{UFSC}
    \membrodabanca{Prof. Dr. Membro da banca 2}{UFSC}
    \membrodabanca{Prof. Dr. Membro da banca 3}{UFSC\,-\,Florianópolis}
\end{folhadeaprovacao}

% 4. Dedicatória
\begin{dedicatoria}
    Este trabalho é dedicado aos meus colegas de classe e aos meus queridos pais.
\end{dedicatoria}

% 5. Agradecimentos 
\begin{agradecimentos}
    Inserir os agradecimentos aos colaboradores à execução do trabalho.
\end{agradecimentos}

% 6. Epígrafe
\begin{epigrafe}
    \aspas
    A natureza é um enorme jogo de xadrez disputado por deuses, e que temos o privilégio de observar.
    As regras do jogo são o que chamamos de física fundamental, e compreender essas regras é a nossa meta.
    \aspas
    \autor{Richard Phillips Feynman} % autor da epígrafe
\end{epigrafe}

% 7. Resumo em português
\begin{resumo}
    O texto do resumo deve ser digitado em um único bloco, sem espaço de parágrafo. Deve ser composto por uma sequência de frases concisas, afirmativas e não de uma enumeração de tópicos. Não deve conter citações. Manter o tempo verbal do texto do trabalho (impessoal) e por vezes usar a voz ativa (Ex.: este trabalho apresenta). O Resumo deve conter: tema, problema, justificativa, objetivos, método e resultados (de forma geral). Abaixo do resumo, informar as palavras-chave (palavras ou expressões significativas retiradas do texto) e preferencialmente não repetir termos do título (para aumentar os indexadores). No mínimo três e no máximo cinco. Separadas por ponto. Observe que o espaçamento aqui, entre linhas, é simples (1,0).

    \palavrachave{Palavra-chave 1}
    \palavrachave{Palavra-chave 2}
    \palavrachave{Palavra-chave 3}
\end{resumo}

% 8a. Resumo em inglês
\begin{resumo}[Abstract]
    \begin{otherlanguage*}{english}
        Resumo traduzido para outros idiomas, neste caso, inglês. Segue o formato do resumo feito na língua vernácula. As palavras-chave traduzidas, versão em língua estrangeira, são colocadas abaixo do texto precedidas pela expressão \emph{Keywords}, separadas por ponto. Observe que o espaçamento aqui, entre linhas, é simples (1,0).
    \end{otherlanguage*}

    \palavrachave{First keyword}
    \palavrachave{Second keyword}
    \palavrachave{Third keyword}
\end{resumo}

% 8b. Resumo em francês
%\begin{resumo}[Résumé]
% \begin{otherlanguage*}{french}
%    Il s'agit d'un résumé en français.
% 
%   \textbf{Mots-clés}: latex. abntex. publication de textes.
% \end{otherlanguage*}
%\end{resumo}
%
%% 8c. Resumo em espanhol
%\begin{resumo}[Resumen]
% \begin{otherlanguage*}{spanish}
%   Este es el resumen en español.
%  
%   \textbf{Palabras clave}: latex. abntex. publicación de textos.
% \end{otherlanguage*}
%\end{resumo}
%% ---

% 9. Lista de figuras
\imprimirlistafiguras

% 10. Lista de quadros
\imprimirlistaquadros

% 11. Lista de tabelas
\imprimirlistatabelas

% 12. Lista de abreviaturas e siglas

% 13. Lista de símbolos 

% 14. Sumário
\imprimirsumario

% ----------------------------------------------------

% Elementos textuais
\textual

% ----------------------------------------------------
\chapter{Introdução}
\label{Chap:intro}

A Introdução é um texto sucinto e direto, no qual deve constar, organizado em parágrafos, com textualidade (coesão e coerência) e sem vícios linguísticos:

\begin{enumerate}[label=\alph*)]
    \item Tema;
    \item Problema;
    \item Justificativa;
    \item Informação das bases/linhas teóricas e/ou autores de referência que serão utilizados no trabalho;
    \item Objetivo geral e, se for o caso, alguns específicos;
    \item Metodologia a ser seguida;
    \item Resultados gerais.
\end{enumerate}

Não crie subtítulos para motivação ou metodologia.
Motivação é justificativa e deve compor o texto.
Para qualquer dúvida, consulte a Norma, não siga exemplos de formatação observados em outros textos, pois podem estar desatualizados ou equivocados.

\section{Objetivos}

Para resolver a problemática X, propõem-se os seguintes objetivos.

\subsection{Objetivo geral}

O objetivo geral deve ser claro, sucinto, direto e coerente com o que foi anunciado no título do trabalho.

\subsection{Objetivos Específicos}

\begin{itemize}
    \item Utilize a lista de verbos indicada para composição de objetivos específicos, conforme  \href{https://www.youtube.com/watch?v=Ycl5a-5gR4w}{disponível no material da disciplina de PTCC};
    \item Os objetivos específicos atingem metas em fases de começo, meio e fim, da pesquisa;
    \item Observe para não colocar a tarefa, mas sim o objetivo que deseja atingir com a mesma.
\end{itemize}

\chapter{Fundamentação teórica} % ou outro título/assunto que for pertinente

Geralmente utilizado para construção da Fundamentação Teórica, este texto introdutório deve apresentar os temas que serão tratados, pode-se resgatar a problemática ou o objetivo geral e fazer uso dos subtítulos para compor esta introdução de capítulo de forma coerente e coesa. Outra opção seria atribuir um título ao capítulo e iniciar o mesmo com o assunto introdutório e só fazer um subtítulo após.

\section{Exposição do tema ou matéria}

Neste capítulo é importante preocupar-se com a estrutura textual (textualidade), que deve manter um padrão de linhas entre os parágrafos, sendo que \href{https://www.youtube.com/watch?v=je0jo6D46w4}{cada parágrafo deve ter duas sentenças} a fim de se manter a coesão e apresentar citações indiretas, principalmente e, por vezes, diretas.
A \href{https://www.youtube.com/watch?v=hIn0_rbKoVw}{pontuação também deve ser foco de atenção}, assim como o cuidado para usar termos do português culto/padrão, em vez de termos coloquiais.

Alguns equívocos são recorrentes em textos acadêmicos:
%
\begin{itemize}
    \item Repetição de termos;
    \item Uso inadequado de ele/ela/dele/dela/nele/nela, já que a voz do texto, como orientado pela Universidade Federal de Santa Catarina (UFSC), é a voz impessoal, dispensando o uso de pronomes;
    \item O uso de siglas deve ser feito como no item anterior, no primeiro uso explique e depois coloque a sigla dentro do parêntese;
    \item O pronome seu/sua, quando utilizado, dispensa o artigo o/a junto, opte pelo artigo ou pelo pronome, preferencialmente pelo artigo;
    \item Afim, assim escrito, é afinidade. A fim como objetivo escreve-se a fim (separado);
    \item Uso de SS e ST em, por ex., esse/desse e neste/deste. Utiliza-se SS quando diz respeito ao que foi dito no texto imediatamente anterior e ST para o que é do presente, como em isso significa e este trabalho, respectivamente;
    \item Use a expressão como ou por exemplo, nunca as duas juntas;
    \item Use itálico somente para \emph{destaques}, é desnecessário colocar itálico em termos em inglês ou latim;
    \item As citações diretas precisam da indicação da página original (ver exemplo da Tabela~\ref{Tab:riscos_porto_sfsul});
    \item As Ilustrações são citações diretas e, da mesma forma, indica-se a página original, com exceções para páginas de internet;
    \item Apêndices são elaborados pelo autor do trabalho, seja qual for a estrutura (textos, tabelas, quadros, fluxogramas, organogramas, roteiros de pesquisa, mapas, listas de cálculo, códigos de programação ou outros), mas que, em função da coesão e harmonia do texto, são destinados aos Apêndices;
    \item Anexos não são elaborados pelo autor do trabalho, seja qual for a estrutura, mas, são considerados importantes para a compreensão do texto, mesmo que, em função da coesão e harmonia do texto, não caibam no corpo textual;
    \item Quadros diferenciam-se de Tabelas pela estrutura e conteúdo. Tabelas têm as linhas externas ausentes e o conteúdo é numérico ou estatístico (Tabela~\ref{Tab:riscos_porto_sfsul}).
          Os Quadros têm todas as linhas, externas e internas e o conteúdo não é numérico.
\end{itemize}

\begin{table}[htb]
    \centering
    \caption{Riscos em cada setor do porto de São Francisco do Sul.}
    \label{Tab:riscos_porto_sfsul}
    \begin{tabular}{lccccc}
        \toprule
        Setor                & Desprezível & Menor & Moderado & Sério & Total \\
        \midrule
        Cais acostável       & 2           & 2     & 4        & 4     & 12    \\
        Pátio de contêineres & 0           & 0     & 5        & 5     & 10    \\
        Vias de circulação   & 0           & 0     & 3        & 4     & 7     \\
        Armazéns             & 0           & 0     & 5        & 2     & 7     \\
        Portões              & 0           & 0     & 3        & 3     & 6     \\
        Sede administrativa  & 0           & 0     & 3        & 1     & 4     \\
        \bottomrule
    \end{tabular}
    \fonte{\citeonline[p. 45]{Conceicao2018}.}
\end{table}

Note-se que, para adequada visualização ou legibilidade da Tabela~\ref{Tab:riscos_porto_sfsul}, a mesma foi reproduzida digitada no texto, assim, conserva-se o nome Tabela, como no original.
Por outro lado, caso seja feito uma imagem da Tabela do trabalho original (um print screen), a ilustração deverá ser nomeada como Figura.

\chapter{Metodologia} % ou Método

Geralmente o terceiro capítulo é utilizado para a apresentação da Metodologia (que é mais do que simplesmente elencar materiais e métodos), capítulo que aponta, a partir do objetivo da pesquisa, a natureza da coleta de dados, o procedimento e o instrumento que será utilizado.
Procure descrever em detalhes os procedimentos da pesquisa.

\chapter{Apresentação de dados}

Início do texto.

\chapter{Análise de dados}

Início do texto.

\chapter{Conclusão}

As conclusões devem responder às questões da pesquisa em relação aos objetivos e às hipóteses.
Deve ser breve, retomar as hipóteses ou objetivo geral, respondendo claramente, objetivamente, sem citações e sem apresentar novos dados.
Finalize indicando recomendações e sugestões para trabalhos futuros.

% ----------------------------------------------------
% Referências
\bibliography{ref.bib}
% ----------------------------------------------------
\postextual

% Apêndice
\begin{apendicesenv}

    \chapter{Título do Apêndice}
    \label{Sec:exemplo_app}

    Textos elaborados pelo autor a fim de complementar a argumentação.
    Deve ser precedido da palavra APÊNDICE, identificada por letras maiúsculas consecutivas, travessão e pelo respectivo título.
    Utilizam-se letras maiúsculas dobradas quando esgotadas as letras do alfabeto.
    Devem estar referenciados no texto, por exemplo: Apêndice~\ref{Sec:exemplo_app}.

    \chapter{Template \LaTeX--UFSC--CTJ}

    % intro
    Esta seção descreve os comandos básicos\footnote{Material introdutório sobre a ferramenta: \href{https://www.overleaf.com/latex/templates/a-quick-guide-to-latex/fghqpfgnxggz}{A quick guide to \LaTeX}.} para a elaboração de um trabalho acadêmico com o template \LaTeX--UFSC--CTJ.
    % visão geral
    Na Seção~\ref{Sec:template_secoes} são demonstrados os comandos para determinar a organização do texto em capítulos, seções, subseções, etc; já na Seção~\ref{Sec:template_referencias} são apresentadas informações acerca do uso das referências, incluindo exemplos para a inclusão de figuras, quadros, tabelas e expressões matemáticas.

    \section{Seções do documento}
    \label{Sec:template_secoes}

    % quadro com a formação e hierarquia das subdivisões
    Este template contém definições para as seções do trabalho, que são organizadas por meio dos comandos indicados no Quadro~\ref{Quad:comandos_secoes}.
    % limite de seções
    Salienta-se que não há um limite máximo ou mínimo de capítulos que podem ser criados; cabe ao autor definir a quantidade que atenda os objetivos do trabalho.

    \begin{quadro}[htb]
        \centering
        \caption{Comandos do \LaTeX~para divisões de capítulo, seção e subseção.}
        \begin{tabular}{|l|l|}
            \hline
            Comando \LaTeX                           & Formatação das seções                  \\ \hline
            \verb|\chapter{Seção primária}|          & \textbf{1. SEÇÃO PRIMÁRIA}             \\ \hline
            \verb|\section{Seção secundária}|        & 1.1. SEÇÃO SECUNDÁRIA                  \\ \hline
            \verb|\subsection{Seção terciária}|      & \textbf{1.1.1. Seção terciária}        \\ \hline
            \verb|\subsubsection{Seção quaternária}| & 1.1.1.1. \underline{Seção quaternária} \\ \hline
            \verb|\subsubsubsection{Seção quinária}| & 1.1.1.1.1. \textit{Seção quinária}     \\ \hline
        \end{tabular}
        \fonte{Autoria própria (\imprimirano).}
        \label{Quad:comandos_secoes}
    \end{quadro}

    % impacto no sumário
    É importante destacar que a definição das seções afeta a construção do sumário do documento.
    % formatação automática
    Note que não é necessário se preocupar com a formatação (caixa alta, negrito ou itálico), uma vez que o template faz as devidas adequações durante a compilação do documento.

    \section{Referências cruzadas}
    \label{Sec:template_referencias}

    % intro
    Nesta seção, são detalhados os principais procedimentos para o uso das referências.
    % conteúdos
    Na Seção~\ref{Sec:ref_cruzadas} são descritos os comandos relacionados às referências cruzadas; as citações de obras são tratadas na Seção~\ref{Sec:ref_biblio}.
    % boas práticas
    Além disso, são apresentadas algumas sugestões de \emph{boas práticas} quanto aos nomes das etiquetas, visando a padronizar algumas nomenclaturas.

    \subsection{Procedimentos gerais}
    \label{Sec:ref_cruzadas}

    % intro
    As referências cruzadas permitem fácil organização e indicação de ilustrações no documento.
    % desenvolvimento
    Para utilizá-las, basta estabelecer uma etiqueta (\verb|\label{citekey}|) e acioná-la com o comando \verb|\ref{citekey}|, onde \verb|citekey| é o nome dado a uma entidade.
    % facilidades
    Afortunadamente, toda numeração é feita de forma automática durante a compilação do documento.
    %  exemplos
    Exemplos de referências cruzadas de Figuras, Quadros, Tabelas e Equações são tratados nas Seções~\ref{Sec:ref_figuras}, \ref{Sec:ref_quadros}, \ref{Sec:ref_tabelas} e \ref{Sec:ref_equacoes}, respectivamente.

    \subsubsection{Figuras}
    \label{Sec:ref_figuras}

    % intro: arquivos de imagem no latex
    Arquivos de imagem são comumente incluídos no documento por meio do comando \verb|\includegraphics|.
    % desenvolvimento: exemplo brasão
    Como exemplo, na Figura~\ref{Fig:brasao_ufsc} temos a imagem em alta qualidade do brasão da universidade, posicionada de forma centralizada.
    % controle do tamanho da figura
    Em relação ao seu tamanho, a ilustração ocupa 15\% da largura do texto, sem que ocorram distorções (a razão de aspecto é mantida).
    % legenda
    Ademais, recomenda-se que as legendas, posicionadas acima da imagem, descrevam de forma específica o seu conteúdo.

    \begin{figure}[htb]
        \centering
        \caption{Brasão da UFSC com fundo claro e sigla.}
        \includegraphics[width=0.15\textwidth]{vertical_sigla_fundo_claro.pdf}
        \fonte{\citeonline{UFSC2023}.}
        \label{Fig:brasao_ufsc}
    \end{figure}

    % ambiente do label
    Observa-se que o comando \verb|\label{citekey}| foi inserido dentro do ambiente da figura; é assim que o compilador consegue estabelecer um contador para a numeração.
    % facilitades relacionadas
    Com o uso do \LaTeX, caso sejam incluídas outras figuras no documento, a numeração é automaticamente atualizada, mantendo-se a consistência das referências cruzadas.

    \subsubsection{Quadros}
    \label{Sec:ref_quadros}

    % "As tabelas e os quadros facilitam a compreensão do fenômeno em estudo, uma vez que apresentam os dados de modo resumido, oferecendo uma visão geral do conteúdo em questão."
    Quadros podem ser utilizados em trabalhos acadêmicos para apresentar informações de forma resumida.
    % organizar dos nomes
    Por exemplo, com o intuito de aprimorar a organização do documento, recomenda-se padronizar os nomes das etiquetas conforme o tipo de entidade, tal como indicado no Quadro~\ref{Quad:sugestao_etiqueta}.
    % formatação
    Ressalta-se que, em termos de formatação, todas as linhas de grade são aparentes.

    \begin{quadro}[htb]
        \centering
        \caption{Sugestão para o uso de etiquetas para as referências cruzadas.}
        \label{tab:my-table}
        \begin{tabular}{|l|l|}
            \hline
            Entidade & Etiqueta             \\ \hline
            Capítulo & \verb|\label{Chap:}| \\ \hline
            Seção    & \verb|\label{Sec:}|  \\ \hline
            Figura   & \verb|\label{Fig:}|  \\ \hline
            Tabela   & \verb|\label{Tab:}|  \\ \hline
            Quadro   & \verb|\label{Quad:}| \\ \hline
            Equação  & \verb|\label{Eq:}|   \\ \hline
        \end{tabular}
        \fonte{Autoria própria (\imprimirano).}
        \label{Quad:sugestao_etiqueta}
    \end{quadro}

    % comentário adicional
    Em conjunto com o texto da legenda, observa-se que as informações apontadas nos Quadros são claras e sucintas, facilitando a compreensão do leitor.
    % fonte
    Assim como no caso das Figuras, é necessário explicitar a fonte do seu conteúdo, com uma indicação logo abaixo da ilustração.

    \subsubsection{Tabelas}
    \label{Sec:ref_tabelas}

    % tabelas
    Diferentemente dos Quadros, as Tabelas apresentam informações numéricas que poderiam compor um gráfico.
    % exemplo
    Na Tabela~\ref{Tab:tensao_deformacao}, temos a descrição do comportamento mecânico de um material hipotético.
    % quadros vs. tabelas
    Portanto, há uma distinção fundamental entre Quadros e Tabelas quanto aos seus conteúdos, refletindo também na forma de apresentação.

    \begin{table}[htb]
        \centering
        \caption{Dados de uma curva tensão-deformação.}
        \begin{tabular}{cc}
            \toprule
            $\sigma$~(MPa) & $\varepsilon$ \\
            \midrule
            0              & 0             \\
            20             & 0,0001        \\
            40             & 0,0002        \\
            \bottomrule
        \end{tabular}
        \fonte{Autoria própria (\imprimirano).}
        \label{Tab:tensao_deformacao}
    \end{table}

    % booktabs
    Neste template, as Tabelas foram geradas com o pacote \verb|booktabs|, que valoriza o uso de linhas horizontais; todavia, é possível optar por utilizar outros formatos, conforme a preferência.
    % tables generator
    De todo modo, para facilitar o preenchimento e organização dos dados, é sugerido o uso da ferramenta online \href{https://www.tablesgenerator.com/}{Tables Generator}.

    \subsubsection{Equações}
    \label{Sec:ref_equacoes}

    % equation
    Expressões incluídas no ambiente matemático são automaticamente destacadas do texto e enumeradas com algarismos arábicos entre parênteses e alinhados à direita.
    % referência
    Para fazer a referência cruzada, é necessário criar uma etiqueta dentro deste ambiente; recomenda-se que o nome (\verb|citekey|) permita uma identificação unívoca do item a ser referenciado, com o intuito de evitar erros.
    % exemplo
    Por exemplo, podemos apresentar a identidade de Euler na expressão
    %
    \begin{equation}
        e^{i \pi} + 1 = 0
        \label{Eq:ident_euler}
    \end{equation}
    %
    e referenciá-la posteriormente por meio da Equação~\ref{Eq:ident_euler}.

    \subsection{Referências}
    \label{Sec:ref_biblio}

    % pacote abntex2cite
    O template \LaTeX-UFSC-CTJ utiliza o pacote \texttt{abntex2cite} para formatar o estilo das referências.
    % funcionamento básico: invocar a citação no texto
    Resumidamente, a depender da situação no texto, é possível fazer a citação de uma obra com o uso de dois comandos principais:
    % 
    \begin{description}
        \item[Citação entre parênteses] \verb|\cite{bibkey}|.
        \item[Citação na frase] \verb|\citeonline{bibkey}|.
    \end{description}
    %
    Onde: \verb|bibkey| é o nome utilizado para identificar a referência, conforme descrito no arquivo \texttt{ref.bib}, que contém as informações das obras.
    % 
    Além disso, é possível agrupar várias referências, separando-as por vírgulas, conforme: \verb|\cite{bibkey_1,bibkey_2}|.
    Já em casos mais específicos, também é possível indicar apenas o ano da publicação: \verb|\citeyear{bibkey}|; ou apenas o seu autor: \verb|\citeauthor{bibkey}| ou ainda \verb|\citeauthoronline{bibkey}|.

    % tipo de documento
    Cada tipo de documento, seja um livro, artigo ou tese, requer o uso de uma entrada específica para a sua devida formatação.
    % resumo
    No Quadro~\ref{Quad:exemplo_bibtex} são apresentados exemplos com os principais tipos de documentos, que devem servir de base para a elaboração das referências\footnote{Consulte o arquivo \texttt{ref.bib} para mais detalhes.}.
    % mais exemplos
    Ademais, uma extensa lista de exemplos pode ser obtida na \href{http://tug.ctan.org/macros/latex/contrib/abntex2/doc/abntex2cite.pdf}{documentação do referido pacote}.

    \begin{quadro}[htb]
        \centering
        \caption{Entradas Bibtex conforme o tipo de documento a ser citado.}
        \begin{tabular}{|l|l|l|}
            \hline
            Tipo de documento              & Entrada Bibtex        & Exemplo (\verb|\citeonline|) \\ \hline
            % Grupo 1. monografia, dissertação, tese, tcc
            Trabalho de conclusão de curso & \verb|@thesis|        & \citeonline{Amaral2016}      \\ \hline
            Dissertação de mestrado        & \verb|@mastersthesis| & \citeonline{Maia2011}        \\ \hline
            Tese de doutorado              & \verb|@phdthesis|     & \citeonline{Gaspar2013}      \\ \hline
            Tese de doutorado (inglês)     & \verb|@thesis|        & \citeonline{May2007}         \\ \hline
            % Grupo 2. livro e artigo de revista científica
            Livro                          & \verb|@book|          & \citeonline{Halliday2023}    \\ \hline
            Artigo de revista científica   & \verb|@article|       & \citeonline{Endo2021}        \\ \hline
            % Grupo 3. trabalhos de congresso
            Trabalho de congresso          & \verb|@inproceedings| & \citeonline{Amador2022}      \\ \hline
            Trabalho de congresso (inglês) & \verb|@inproceedings| & \citeonline{Leahy2019}       \\ \hline
            % Grupo 4. normas técnicas e manuais
            Normas técnicas                & \verb|@manual|        & \citeonline{NBR6023}         \\ \hline
            Relatório do governo           & \verb|@techreport|    & \citeonline{MEC1996}         \\ \hline
        \end{tabular}
        \fonte{Autoria própria (\imprimirano).}
        \label{Quad:exemplo_bibtex}
    \end{quadro}

    % problema envolvendo letras acentuadas escritas em maiúsculo
    Como mencionado no
    \href{https://github.com/abntex/abntex2/wiki/FAQ#problemas-com-convers%C3%A3o-para-mai%C3%BAsculas}{FAQ do \texttt{abntex2cite}}, infelizmente ainda é necessário gerar caracteres especiais mediante comandos do \LaTeX\footnote{Uma lista de comandos relacionados a caracteres especiais pode ser obtida  \href{https://en.wikibooks.org/wiki/LaTeX/Special_Characters}{neste link}.} para a devida formatação das letras em maiúsculo com acentos.
    % Por exemplo: utilizar o comando \verb|{\^{o}}| para gerar {\^{o}}, \verb|{\~{a}}| para gerar {\~{a}}, e assim por diante.
    Especificamente, deve-se atentar para os campos que podem ser formatados em caixa alta, como autor e nome do congresso.
    Por exemplo: utilizar \verb|author={C. E. Concei{\c{c}}{\~a}o}|, ao invés de \verb|author={C. E. Conceição}|.

    % citações longas e citações diretas.
    Finalmente, as citações longas podem ser feitas por meio do ambiente \verb|quoting|, conforme exemplificado a seguir:
    %
    \begin{quoting}
        Após a ilustração, na parte inferior, indicar a fonte consultada (elemento obrigatório, mesmo que seja produção do próprio autor), legenda, notas e outras informações necessárias à sua compreensão (se houver).
        A ilustração deve ser citada no texto e inserida o mais próximo possível do texto a que se refere. \cite[p. 11]{NBR14724}.
    \end{quoting}
    %
    Destaca-se o emprego da citação direta, como também observado na Tabela~\ref{Tab:riscos_porto_sfsul}. Para incluir o número da página junto com referência, basta fazer uma indicação no campo opcional (entre colchetes), conforme exemplo: \verb|\cite[p. x]{bibkey}|.

    Recomendações adicionais:
    %
    \begin{itemize}
        \item As normas ABNT não precisam conter o endereço de internet (url).
        \item Não é necessário inserir o ISBN no bibtex.
        \item No caso de livros, incluir a cidade da editora no campo \verb|address|.
    \end{itemize}

\end{apendicesenv}

% ----------------------------------------------------
% Anexo
\begin{anexosenv}

    \chapter{Título do anexo}
    \label{Sec:exemplo_ane}

    São documentos não elaborados pelo autor que servem como fundamentação (mapas, leis, estatutos).
    Deve ser precedido da palavra ANEXO, identificada por letras maiúsculas consecutivas, travessão e pelo respectivo título.
    Utilizam-se letras maiúsculas dobradas quando esgotadas as letras do alfabeto.
    Devem estar referenciados no texto, por exemplo: Anexo~\ref{Sec:exemplo_ane}.

\end{anexosenv}

\end{document}
