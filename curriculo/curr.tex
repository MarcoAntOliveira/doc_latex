\documentclass[30.0pt a4paper]{moderncv}

% Temas disponíveis: 'casual', 'classic', 'oldstyle' e 'banking'
\moderncvstyle{classic}

% Cores disponíveis: 'blue', 'orange', 'green', 'red', 'purple', 'grey' e 'black'
\moderncvcolor{blue}

% Informações pessoais
\firstname{Marcos Antonio}
\familyname{Tomé Oliveira}
\title{}
\address{Rua Albatroz 518}{Joinville, Santa Catarina}
\phone{(21) 97219-2303}
\email{marcoolivera096@gmail.com}

% Seu conteúdo do currículo vai aqui
\begin{document}
\makecvtitle
\gitlabsocialsymbol{https://github.com/MarcoAntOliveira}
% Seções do currículo
\section{Formação Acadêmica}
\cventry{2011--2014}{Ensino médio}{C.E.M Maria Veralba ferraz}{Papucaia}{\textit{}}{}
\cventry{2022--atualmente}{Ensino Superior}{Universidade Federal de Santa Catarina}{Joinville}{\textit{}}{Engenharia mecatronica }

\section{Andamento}
\cventry{2023--atualmente}{SolidWorks}{Udemy}{online}{}{curso na area de modelagem de moldes}

\cventry{2023- atualmente}{Sistemas Embarcados}{Udemy}{online}{}{
    \begin{itemize}
        \item Interrupções e timers
        \item relação de complexidade x performance e protoclos de comunicação
        \item  padrão de codificação , macros/defines , modularização de código e condicionais escolha de um microcontrolador , modalidade de siftware embarcado e dicas
        \item Maquinas de estados e variaveis =
    \end{itemize}
}

\cventry{2023- atualmente}{Free Rtos e esp32}{Udemy}{online}{}{
    \begin{itemize}
        \item  Tasks noFreeRTOS
        \item Filas  
        \item Semáforos
        \item Softwares timers
        \item Event Groups
        \item Task Notifications 
    \end{itemize}
}

\section{Experiência Profissional}
\cventry{2014--2020}{Garçom}{Empresa}{Cachoeiras de macacu}{}{Além de garçom também era responsavel pelo caixa em e partes por sua gerência}

\section{Habilidades}
\cvitem{Docker}{
    \begin{itemize}
        \item Básico de docker
        \item Linux
    \end{itemize}
}

\cvitem{python}
{
    \begin{itemize}
        \item básico de programação e de programação em Python
        \item programação orinetada a objetos
        \item intefaces gráficas com PySide 6
        \item banco de dados
        \item frameworks gerais com django
        \item design patterns
        \item banco de dados relacionais
    \end{itemize}
}
\cvitem{inglês Intermediário}{
    \textbf { UFSC $-$ Florianópolis }
    }
\cvitem{Latex}{
    \begin{itemize}
        \item produção de artigos de acordo com a ABNT
        \item Documentação de codigos
        \item produção de relátorios e curriculos 
    \end{itemize}
}
\cvitem{Git e Github}{
    \begin{itemize}
        \item criar repositorios
        \item usar como servidor 
        \item subir API para o repositorio
    \end{itemize}
    }
\section{Disponibilidade}
\cvitem{}{disponibildade para trabalhar e viajar aos finais de semanas}


\end{document}
