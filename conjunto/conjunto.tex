\documentclass[a4paper, 16pt]{}
\usepackage[top=2cm, bottom=2cm, left=2.5cm, right=2.5cm]{geometry}
\usepackage[utf8]{inputenc}
\usepackage{amsmath, amsfonts, amssymb}
\date{} % Remove a exibição da data
\usepackage{xcolor}
\usepackage{listings}
\usepackage{graphicx}

\lstset{
  language=Python,
  basicstyle=\ttfamily,
  keywordstyle=\bfseries\color{blue},
  commentstyle=\color{red},
  stringstyle=\color{red!70!black},
  numberstyle=\tiny,
  stepnumber=1,
  numbersep=5pt,
  backgroundcolor=\color{white},
  breaklines=true,
  breakautoindent=true,
  showspaces=false,
  showstringspaces=false,
  showtabs=false,
  tabsize=2
}
\title{Python}
\begin{document}
\maketitle
\section{Variaveis}
Python = Linguagem programação\\
tipo de tipagem = Dinamica/Forte\\
str $->$ string $->$ texto\\
Strings são textos que estão dentro das aspas


\section{Coerção}
conversão de tipos , coerção
type convertion, typecasting, coercion
é o ato de converter um tipo em outro
tipos imutaveis e primitivos
str, int , float, bool 

\begin{lstlisting}


\end{lstlisting}

\section{Separador}
nesse trecho de codigo  sep indica que os numeros inteiros seram separados
por - e ao final e o end barra n indica que havera uma quebra de linha
\begin{lstlisting}
print(12, 34, 1011, sep= "-", end='\n##')
print(9, 10, sep= "-", end= '\n')
print(56, 78, sep= '-', end='\n')
\end{lstlisting}

\section{F-strings}
consigo inserir dentro de uma string as variaveis dentro do codigo
\begin{lstlisting}
  nome = 'Luiz Otavio'
  altura = 1.80
  peso = 95
  imc = peso / altura ** 2
  
  "f-strings"
  linha_1 = f'{nome} tem {altura:.2f} de altura,'
  linha_2 = f'pesa {peso} quilos e seu imc e'
  linha_3 = f'{imc:.2f}'
  
  print(linha_1)
  print(linha_2)
  print(linha_3)
  
  # Luiz Otavio tem 1.80 de altura,
  # pesa 95 quilos e seu IMC e
  # 29.320987654320987
\end{lstlisting}  

\section{Formatacao strings usando format}  
setando a quantidade de casas apos a virgula
\begin{lstlisting}
  a = 'AAAAA'
  b = 'BBBBBB'
  c = 1.1
  string = 'b={nome2} a={nome1} a={nome1} c={nome3:.2f}'
  formato = string.format(
      nome1=a, nome2=b, nome3=c
  )
  
  print(formato)

\end{lstlisting}


\section{Operadores logicos}
     Operadores logicos
and (e) or (ou) not (não)\\
 and - Todas as condições precisam ser
 verdadeiras.
 Se qualquer valor for considerado falso,
 a expressão inteira será avaliada naquele valor
 São considerados falsy (que vc já viu)
 0 0.0 '' False
 Também existe o tipo None que é
 usado para representar um não valor\\
 \subsection{exemplo and}
\begin{lstlisting}
entrada = input('[E]ntrar [S]air: ')
senha_digitada = input('Senha: ')

senha_permitida = '123456'

 if entrada == 'E' and senha_digitada == senha_permitida:
     print('Entrar')
 else:
     print('Sair')
#Avaliacao de curto circuito
print(True and False and True)
print(True and 0 and True)
\end{lstlisting}
\subsection{exemplo or}
esse exemplo serve para verificar senha ou se não há senha
\begin{lstlisting}
# entrada = input('[E]ntrar [S]air: ')
# senha_digitada = input('Senha: ')

# senha_permitida = '123456'

# if (entrada == 'E' or entrada == 'e') and senha_digitada == senha_permitida:
#     print('Entrar')
# else:
#     print('Sair')

# Avaliacao de curto circuito
senha = input('Senha: ') or 'Sem senha'
print(senha)
\end{lstlisting} 
\subsection{operador not}
usado para inverter expressoes convem as vezes usar dentro de um print 
\begin{lstlisting}
  @@ -0,0 +1,7 @@
# Operador logico "not"
# Usado para inverter expressoes
# not True = False
# not False = True
# senha = input('Senha: ')
print(not True)  # False
print(not False)  # True
\end{lstlisting} 
\subsection{operador not in}

\begin{lstlisting}
# Operadores in e not in
# Strings sao iteraveis
#  0 1 2 3 4 5
#  O t a v i o
# -6-5-4-3-2-1
# nome = 'Otavio'
# print(nome[2])
# print(nome[-4])
# print('vio' in nome)
# print('zero' in nome)
# print(10 * '-')
# print('vio' not in nome)
# print('zero' not in nome)

nome = input('Digite seu nome: ')
encontrar = input('Digite o que deseja encontrar: ')

if encontrar in nome:
    print(f'{encontrar} esta em {nome}')
else:
    print(f'{encontrar} nao esta em {nome}')
\end{lstlisting}

\section{interpolação de string com porcentagem em python}
\begin{lstlisting}
  """
Interpolacao basica de strings
s - string
d e i - int
f - float
x e X - Hexadecimal (ABCDEF0123456789)
"""
nome = 'Luiz'
preco = 1000.95897643
variavel = '%s, o preco e R$%.2f' % (nome, preco)
print(variavel)
#conversao de inteiro decimal para hexadecimal
print('O hexadecimal de %d e %08X' % (1500, 1500))

\end{lstlisting}
\section{Formatacao de strings com Fstrings}
\begin{lstlisting}
  """
  Formatacao basica de strings
  s - string
  d - int
  f - float
  .<numero de digitos>f
  x ou X - Hexadecimal
  (Caractere)(><^)(quantidade)
  > - Esquerda
  < - Direita
  ^ - Centro
  = - Forca o numero a aparecer antes dos zeros
  Sinal - + ou -
  Ex.: 0>-100,.1f
  Conversion flags - !r !s !a 
  """
  variavel = 'ABC'
  print(f'{variavel}')
  print(f'{variavel: >10}')
  print(f'{variavel: <10}.')
  print(f'{variavel: ^10}.')
  print(f'{1000.4873648123746:0=+10,.1f}')
  print(f'O hexadecimal de 1500 e {1500:08X}')
  print(f'{variavel!r}')  
\end{lstlisting}

\end{document}