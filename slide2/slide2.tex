\usepackage{enumitem}
\begin{document}
%\documentclass{beamer}
% Tema da apresentação
%\usetheme{Antibes}
A classe `beamer` no LaTeX oferece uma variedade de temas para personalizar a aparência das apresentações. Aqui estão alguns dos temas disponíveis.

1. **AnnArbor**: Um tema simples com cores suaves e uma barra lateral para títulos de seções.

2. **Antibes**: Um tema que inclui uma barra lateral com miniaturas de slides e uma navegação mais visual.

3. **Bergen**: Um tema que divide os slides em duas partes, uma para o conteúdo principal e outra para uma barra lateral de navegação.

4. **Berkeley**: Apresenta uma barra lateral colorida e títulos de seções destacados.

5. **Berlin**: Oferece uma barra de navegação no topo e uma aparência moderna.

6. **Boadilla**: Um tema limpo com uma barra inferior de informações.

7. **CambridgeUS**: Um tema com um esquema de cores azul e branco e uma barra superior.

8. **Copenhagen**: Possui um design minimalista com uma barra de navegação no topo.

9. **Darmstadt**: Tem uma barra de navegação no topo e uma barra lateral para títulos de seções.

10. **Dresden**: Possui uma aparência simples com títulos de seções em uma barra lateral.

11. **Frankfurt**: Inclui uma barra de navegação no topo e títulos de seções na barra lateral.

12. **Goettingen**: Tem uma barra lateral para títulos de seções e miniaturas de slides na parte inferior.

13. **Hannover**: Oferece uma barra superior de navegação e títulos de seções na lateral.

14. **Ilmenau**: Um tema que coloca títulos de seções em uma faixa colorida no topo.

15. **JuanLesPins**: Possui uma barra lateral de navegação e um visual moderno.

16. **Luebeck**: Inclui uma barra superior de navegação e títulos de seções na lateral.

17. **Madrid**: Um tema padrão com uma barra superior de navegação e títulos de seções destacados.

18. **Malmoe**: Possui uma barra lateral para títulos de seções e uma aparência limpa.

19. **Marburg**: Oferece títulos de seções em uma barra lateral e miniaturas de slides na parte inferior.

20. **Montpellier**: Apresenta uma barra de navegação no topo e uma aparência moderna.

21. **PaloAlto**: Tema com títulos de seções na barra lateral e uma barra inferior de navegação.

22. **Pittsburgh**: Inclui uma barra lateral com miniaturas de slides e títulos de seções destacados.

23. **Rochester**: Apresenta títulos de seções em uma barra lateral e miniaturas de slides na parte inferior.

24. **Singapore**: Possui uma barra de navegação no topo e uma barra lateral para títulos de seções.

25. **Szeged**: Tema com uma barra de navegação no topo e títulos de seções na barra lateral.

26. **Warsaw**: Um tema com uma barra superior de navegação e uma aparência elegante.

Lembre-se de que você pode experimentar diferentes temas e ajustar as configurações para atender às suas preferências. Cada tema oferece um estilo visual único para suas apresentações.
% Título, autor e instituição
\title{Título da Apresentação}
\author{Seu Nome}
\institute{Nome da Instituição}
\date{\today}

\begin{document}

\begin{frame}
  \titlepage
\end{frame}

\begin{frame}{Índice}
  \tableofcontents
\end{frame}

\section{Introdução}
\begin{frame}{Introdução}
  Esta é a introdução da apresentação.
\end{frame}

\section{Desenvolvimento}
\begin{frame}{Primeiro Tópico}
  Este é o conteúdo do primeiro tópico.
\end{frame}

\begin{frame}{Segundo Tópico}
  Este é o conteúdo do segundo tópico.
\end{frame}

\section{Conclusão}
\begin{frame}{Conclusão}
  Esta é a conclusão da apresentação.
\end{frame}

\end{document}
