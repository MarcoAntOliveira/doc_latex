\documentclass{beamer}

% Tema da apresentação (opcional)
% Existem diversos temas disponíveis, você pode escolher o que melhor se adequa à sua apresentação
%\usetheme{default}
\usetheme{Warsaw}

% Defina cores personalizadas
\definecolor{cor_fundo}{RGB}{230,230,230}
\definecolor{cor_titulo}{RGB}{0,0,0}
\definecolor{cor_blocos}{RGB}{100,150,50}

% Personalize as cores do tema
\setbeamercolor{background canvas}{bg=cor_fundo} % Fundo dos slides
\setbeamercolor{title}{fg=cor_titulo} % Cor do título
\setbeamercolor{block title}{bg=cor_blocos, fg=blue} % Título dos blocos
\setbeamercolor{block body}{bg=cor_fundo, fg=black} % Corpo dos blocos
\setbeamercolor{item}{fg=cor_blocos} % Cor dos itens


% Título, autor e data da apresentação
\title{Escalonamento - Sistemas Operacionais}
\author{Seu Nome}
\date{\today}

\begin{document}

% Slide de título
\begin{frame}
    \titlepage
\end{frame}

% Slide de conteúdo
\begin{frame}
    \frametitle{Introdução}
    Este é um slide de conteúdo.
    \begin{itemize}
        \item Item 1
        \item Item 2
        \item Item 3
    \end{itemize}
\end{frame}

% Outro slide de conteúdo
\begin{frame}
    \frametitle{Mais conteúdo}
    \begin{enumerate}
        \item Primeiro item numerado
        \item Segundo item numerado
    \end{enumerate}
\end{frame}

% Slide de conclusão
\begin{frame}
    \frametitle{Conclusão}
    \begin{block}{Resumo}
        Esta é a conclusão da apresentação.
    \end{block}
    
    \begin{alertblock}{Importante}
        Lembre-se de que o LaTeX é uma excelente ferramenta para criar slides bem estruturados e profissionais.
    \end{alertblock}
\end{frame}

\end{document}
