\documentclass[16pt]{article}
\usepackage[a4paper, left=1 cm, right=1cm, top=1.5cm, bottom=1cm] {geometry}
\date{}
\title{SolidWorks}

\usepackage{setspace}
\begin{document}

\maketitle
Esse relatorio visa descrever os comandos passados no curso de solidworks
\section{Definir esboços}
Definir um esboço, relacionando os componentes clicando com o botao esquerdo do mouse , em seguida um clique com o botao esquedo  mouse  + CTRL na entidade que se deseja 
\section{Unidades de medidas}
No canto inferior direito MMGS ao clicar com botão esquerdo é capaz de alterar o conjunto de unidades de medidas
\section{atalhos e obs importantes}
CRTL + shift seleciona uma linha inteira e não só o ponto.
usar o arco de três pontos pode ser uma alternativa para casos onde há necessidade de cantos arredondados
primeiro os dois pontos da extremidade dps o centro


\end{document}
