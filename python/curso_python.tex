\documentclass[16pts]{article}
\usepackage[a4paper, left=1 cm, right=1cm, top=0.5cm, bottom=1cm] {geometry}
\date{} % Remove a exibição da data
\usepackage{xcolor}
\usepackage{listings}
\usepackage{graphicx}
\usepackage[utf8]{inputenc}
\usepackage[T1]{fontenc}

\lstset{
  language=C,
  basicstyle=\ttfamily,
  keywordstyle=\bfseries\color{blue},
  commentstyle=\color{red},
  stringstyle=\color{blue!70!black},
  numberstyle=\tiny,
  stepnumber=1,
  numbersep=5pt,
  backgroundcolor=\color{white},
  breaklines=true,
  breakautoindent=true,
  showspaces=false,
  showstringspaces=false,
  showtabs=false,
  tabsize=2
}
\title{Python}
\begin{document}
\maketitle
\section{Variaveis}
Python = Linguagem programação\\
tipo de tipagem = Dinamica/Forte\\
str $->$ string $->$ texto\\
Strings são textos que estão dentro das aspas
\section{Coerção}
conversão de tipos , coerção
type convertion, typecasting, coercion
é o ato de converter um tipo em outro
tipos imutaveis e primitivos
str, int , float, bool 

\begin{lstlisting}
  # conversao de tipos, coercao
# type convertion, typecasting, coercion
# e o ato de converter um tipo em outro
#  tipos imutaveis e primitivos:
# str, int, float, bool
print(int('1'), type(int('1')))
print(type(float('1') + 1))
print(bool(' '))
print(str(11) + 'b')


\end{lstlisting}
\section{Separador}
nesse trecho de codigo  sep indica que os numeros inteiros seram separados
por - e ao final e o end barra n indica que havera uma quebra de linha
\begin{lstlisting}
print(12, 34, 1011, sep= "-", end='\n##')
print(9, 10, sep= "-", end= '\n')
print(56, 78, sep= '-', end='\n')
\end{lstlisting}
\section{F-strings}
consigo inserir dentro de uma string as variaveis dentro do codigo
\begin{lstlisting}
  nome = 'Luiz Otavio'
  altura = 1.80
  peso = 95
  imc = peso / altura ** 2
  
  "f-strings"
  linha_1 = f'{nome} tem {altura:.2f} de altura,'
  linha_2 = f'pesa {peso} quilos e seu imc e'
  linha_3 = f'{imc:.2f}'
  
  print(linha_1)
  print(linha_2)
  print(linha_3)
  
  # Luiz Otavio tem 1.80 de altura,
  # pesa 95 quilos e seu IMC e
  # 29.320987654320987
\end{lstlisting}  
\section{Formatacao strings usando format}  
setando a quantidade de casas apos a virgula
\begin{lstlisting}
  a = 'AAAAA'
  b = 'BBBBBB'
  c = 1.1
  string = 'b={nome2} a={nome1} a={nome1} c={nome3:.2f}'
  formato = string.format(
      nome1=a, nome2=b, nome3=c
  )
  
  print(formato)

\end{lstlisting}
\section{Operadores logicos}
obs: print sempre avalia true por exemplo print (nome and idade) se não houver nada nas duas variaveis 
a expressão retornar false, se houvera expressão retorna false
and (e) or (ou) not (não)\\
 and - Todas as condições precisam ser
 verdadeiras.
 Se qualquer valor for considerado falso,
 a expressão inteira será avaliada naquele valor
 São considerados falsy (que vc já viu)
 0 0.0 '' False
 Também existe o tipo None que é
 usado para representar um não valor\\
  \subsection{exemplo and}
  \begin{lstlisting}
  entrada = input('[E]ntrar [S]air: ')
  senha_digitada = input('Senha: ')

  senha_permitida = '123456'

  if entrada == 'E' and senha_digitada == senha_permitida:
      print('Entrar')
  else:
      print('Sair')
  #Avaliacao de curto circuito
  print(True and False and True)
  print(True and 0 and True)
  \end{lstlisting}
  \subsection{exemplo or}
  esse exemplo serve para verificar senha ou se não há senha
  \begin{lstlisting}
  # entrada = input('[E]ntrar [S]air: ')
  # senha_digitada = input('Senha: ')

  # senha_permitida = '123456'

  # if (entrada == 'E' or entrada == 'e') and senha_digitada == senha_permitida:
  #     print('Entrar')
  # else:
  #     print('Sair')

  # Avaliacao de curto circuito
  senha = input('Senha: ') or 'Sem senha'
  print(senha)
  \end{lstlisting} 
  \subsection{operador not}
  usado para inverter expressoes convem as vezes usar dentro de um print 
  \begin{lstlisting}
    @@ -0,0 +1,7 @@
  # Operador logico "not"
  # Usado para inverter expressoes
  # not True = False
  # not False = True
  # senha = input('Senha: ')
  print(not True)  # False
  print(not False)  # True
  \end{lstlisting} 
  \subsection{operador not in}
\begin{lstlisting}
# Operadores in e not in
# Strings sao iteraveis
#  0 1 2 3 4 5
#  O t a v i o
# -6-5-4-3-2-1
# nome = 'Otavio'
# print(nome[2])
# print(nome[-4])
# print('vio' in nome)
# print('zero' in nome)
# print(10 * '-')
# print('vio' not in nome)
# print('zero' not in nome)

nome = input('Digite seu nome: ')
encontrar = input('Digite o que deseja encontrar: ')

if encontrar in nome:
    print(f'{encontrar} esta em {nome}')
else:
    print(f'{encontrar} nao esta em {nome}')
\end{lstlisting}
\section{operadores aritmeticos}
  \begin{lstlisting}
    adicao = 10 + 10
    print('Adicao', adicao)
    
    subtracao = 10 - 5
    print('Subtracao', subtracao)
    
    multiplicacao = 10 * 10
    print('Multiplicacao', multiplicacao)
    
    divisao = 10 / 3  # float
    print('Divisao', divisao)
    
    divisao_inteira = 10 // 3
    print('Divisao inteira', divisao_inteira)
    
    exponenciacao = 2 ** 10
    print('Exponenciacao', exponenciacao)
    
    modulo = 55 % 2  # resto da divisão
    print('Modulo', modulo)
    
    print(10 % 8 == 0)
    print(16 % 8 == 0)
    print(10 % 2 == 0)
    print(15 % 2 == 0)
    print(16 % 2 == 0)
  \end{lstlisting}
\section{interpolação de string com porcentagem em python}
\begin{lstlisting}
  """
s - string
d e i - int
f - float
x e X - Hexadecimal (ABCDEF0123456789)
"""
nome = 'Luiz'
preco = 1000.95897643
variavel = '%s, o preco e R$%.2f' % (nome, preco)
print(variavel)
#conversao de inteiro decimal para hexadecimal
print('O hexadecimal de %d e %08X' % (1500, 1500))

\end{lstlisting}
\section{Formatção de strings com F-strings}
\begin{lstlisting}
  """
  s - string
  d - int
  f - float
  .<numero de digitos>f
  x ou X - Hexadecimal
  (Caractere)(><^)(quantidade)
  > - Esquerda
  < - Direita
  ^ - Centro
  = - Forca o numero a aparecer antes dos zeros
  Sinal - + ou -
  Ex.: 0>-100,.1f
  Conversion flags - !r !s !a 
  """
  variavel = 'ABC'
  print(f'{variavel}')
  print(f'{variavel: >10}')
  print(f'{variavel: <10}.')
  print(f'{variavel: ^10}.')
  print(f'{1000.4873648123746:0=+10,.1f}')
  print(f'O hexadecimal de 1500 e {1500:08X}')
  print(f'{variavel!r}')
\end{lstlisting}
\section{Fatiamento de strings}
\begin{lstlisting}
  """
 012345678
 Ola mundo
-987654321
para que o fatiamento aconteca ate o final deve ser o indice final + 1
ou nao ter nada
Fatiamento [i:f:p] [::]
Obs.: a funcao len retorna a qtd 
de caracteres da str

"""
variavel = 'Ola mundo'
print(variavel[::-1])
#contagem de tras pra frente -1 indica os passos , -1 indica de onde #comeca e -10 onde termina
#obs: o numero de passos pode ser maior que um  
print(variavel[-1:-10:-1])
\end{lstlisting}  
\section{Introdução a Try e Except}
\begin{lstlisting}
  """
try -> tentar executar o codigo
except -> ocorreu algum erro ao tentar executar
"""
numero_str = input(
    'Vou dobrar o numero que vc digitar: '
)

try:
    numero_float = float(numero_str)
    print('FLOAT:', numero_float)
    print(f'O dobro de {numero_str} e {numero_float * 2:.2f}')
except:
    print('Isso nao e um numero')

# if numero_str.isdigit():
#     numero_float = float(numero_str)
#     print(f'O dobro de {numero_str} e {numero_float * 2:.2f}')
# else:
#     print('Isso nao e um numero')
\end{lstlisting}
\section{id - A identidade do valor que está na memoria}
entre duas variaveis na memoria com o mesmo valor , o python sempre mostra a primeira variavel endereçada na  memória.
\begin{lstlisting}
  #exemplo id
  v1 = 'a'
  v2 = 'a'
  #printa mesmo id de v1 devido o mesmo valor da memoria, se v1!=v2 sera printado dois valores diferentes de id 

  print(id(v1))
  print(id(v2))

\end{lstlisting}
\section{Flags, is, is not, e None}
\begin{lstlisting}
  """
Flag (Bandeira) - Marcar um local
None = Nao valor
is e is not = e ou nao e (tipo, valor, identidade)
id = Identidade
"""
condicao = False
passou_no_if = None

if condicao:
    passou_no_if = True
    print('Faca algo')
else:
    print('Nao faca algo')


if passou_no_if is None:
    print('Nao passou no if')
else:
    print('Passou no if')
  
\end{lstlisting}
\section{Tipos built-in, tipos imutaveis, metodos de strings, documentação}
\begin{lstlisting}
  """
https://docs.python.org/pt-br/3/library/stdtypes.html
Imutaveis que vimos: str, int, float, bool
"""
Tipos Built-in = tipos imbutidos
Tipos imutaveis =  nao podem acontecer mudancas no tipo
string = '1000'
# outra_variavel = f'{string[:3]}ABC{string[4:]}'
# print(string)
# print(outra_variavel)
print(string.zfill(10))
  
\end{lstlisting}
\section{while}
  \begin{lstlisting}
    """
  Repeticoes
  while (enquanto)
  Executa uma acao enquanto uma condicao for verdadeira
  Loop infinito -> Quando um codigo nao tem fim
  """
  condicao = True

  while condicao:
      nome = input('Qual o seu nome: ')
      print(f'Seu nome e {nome}')

      if nome == 'sair':
          break
  print('Acabou')
  \end{lstlisting}
\section{operadores de atribuição com operadores aritmeticos}
  \begin{lstlisting}
    """
    Operadores de atribuicao
    = += -= *= /= //= **= %=
    """
    contador = 10
    
    ###
    
    contador /= 5
    print(contador) 
  \end{lstlisting}
\section{while + continue pulando repeticoes}
  \begin{lstlisting}
    """
    Repeticoes
    while (enquanto)
    Executa uma acao enquanto uma condicao for verdadeira
    Loop infinito -> Quando um codigo nao tem fim
    """
    contador = 0
    
    while contador <= 100:
        contador += 1
    
        if contador == 6:
            print('Nao vou mostrar o 6.')
            continue
    
        if contador >= 10 and contador <= 27:
            print('Nao vou mostrar o', contador)
            continue
    
        print(contador)
    
        if contador == 40:
            break
    
    
    print('Acabou')
  \end{lstlisting}
\section{while + while(lacos internos)} 
  \begin{lstlisting}
    """
    Repetições
    while (enquanto)
    Executa uma ação enquanto uma condição for verdadeira
    Loop infinito -> Quando um código não tem fim
    """
    qtd_linhas = 5
    qtd_colunas = 5
    
    linha = 1
    while linha <= qtd_linhas:
        coluna = 1
        while coluna <= qtd_colunas:
            print(f'{linha=} {coluna=}')
            coluna += 1
        linha += 1
    
    
    print('Acabou') 
  \end{lstlisting}
\section{while-else}
o else é sempre executado apos o while, porém se houver um break, o else não é executado
\begin{lstlisting}
  """ while/else """
string = 'Valor qualquer'

i = 0
while i < len(string):
    letra = string[i]

    if letra == ' ':
        break

    print(letra)
    i += 1
else:
    print('Nao encontrei um espaco na string.')
print('Fora do while.')
\end{lstlisting}
\section{for-in}
\begin{lstlisting}
  # senha_salva = '123456'
# senha_digitada = ''
# repeticoes = 0

# while senha_salva != senha_digitada:
#     senha_digitada = input(f'Sua senha ({repeticoes}x): ')

#     repeticoes += 1

# print(repeticoes)
# print('Aquele laco acima pode ter repeticoes infinitas')
texto = 'Python'

novo_texto = ''
for letra in texto:
    novo_texto += f'*{letra}'
    print(letra)
print(novo_texto + '*')
\end{lstlisting}
\section{for + range}
\begin{lstlisting}
  """
For + Range
range -> range(start, stop, step)
"""
numeros = range(0, 100, 8)

for numero in numeros:
    print(numero)
\end{lstlisting}
\section{for por baixo dos panos}
\begin{lstlisting}
  """
Iteravel -> str, range, etc (__iter__)
Iterador -> quem sabe entregar um valor por vez
next -> me entregue o proximo valor
iter -> me entregue seu iterador
"""
# for letra in texto
texto = 'Luiz'  # iteravel

# iteratador = iter(texto)  # iterator

# while True:
#     try:
#         letra = next(iteratador)
#         print(letra)
#     except StopIteration:
#         break

for letra in texto:
    print(letra)
\end{lstlisting}
\section{for e suas variações}
\begin{lstlisting}
  for i in range(10):
  if i == 2:
      print('i e 2, pulando...')
      continue

  if i == 8:
      print('i e 8, seu else nao executara')
      break

  for j in range(1, 3):
      print(i, j)
else:
  print('For completo com sucesso!') 
\end{lstlisting}
\end{document}
